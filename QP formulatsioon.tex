\documentclass[twosided, 11pt, a4paper]{article}
\usepackage[estonian]{babel}
\usepackage{amsmath}
\usepackage{titling}

\newcommand{\subtitle}[1]{%
  \posttitle{%
    \par\end{center}
    \begin{center}\large#1\end{center}
    \vskip0.5em}%
}

\title{Tootmisvarade Optimeerimine}
\subtitle{Eesti Energia, Energiakaubandus}
\author{Allan Puusepp \\ Taaniel Uleksin \\ Peeter Meos }

\begin{document}

\maketitle

\section{\"Ulesande p\"ustitus}
Eesti Energia otsib  lahendust \"ulesandele toota aastases perioodis tunnise t\"psusega kasumit maksimeerival viisil elektrit. Primaarseteks energiakandjateks on erinevate kaevanduste erinevate omadustega p\~olevkivi, millest toodavad elektrit erinevad elektrijaamad. Elektrijaamade kasutegurid on teada, kulufunktsioon on laias laastus teada koos lisanduvate saastekuludega. Tulufunktsioon on \"uldjoontes teada, aga on veel formuleerimisel.
\subsection{Teadaolevad piirangud}
\begin{enumerate}
\item Iga elektrijaamal on maksimumv\~oimsus, mis on ajaliselt muutuv. Sel viisil on v\~oimalik modelleerida hooldust ja \emph{downtime} m\~oju.
\item Elektrijaama v\~oimsust ei saa suurendada ega v\"ahendada rohkem kui etteantud $\Delta$ \emph{MW} tunnis. 
\item Igal kaevandusel ja seega igal k\"utuset\"u\"ubil on \"ulemine mahupiir, millest rohkem vastavat k\"utust toota ei ole v\~oimalik.
\item Elektritootmine peab olema tagatud etteantud miinimumv\~oimsustel.
\end{enumerate}

\subsection{Katmata alad}
Antud hetkel ja k\"aesolevas formuleeringus ei sisaldu \~olitootmisega seotud tulusid ja kulusid. J\"argmistes iteratsioonides peaks optimeerimismudel sisaldama peale elektritootmise ka \~olitootmist ning ilmselt lisama piirangu, milles elektritootmine ei tohi minna nulli, st. k\~oik elektrienergia ostetakse sisse ja Eesti Energia keskendub \~olitootmisele.

\section{Optimeerimis\"ulesande matemaatiline formulatsioon}
\begin{align}
\mathrm{max}\; \prod &= \sum_{i=1}^n P(x_i) - \sum_{i=1}^n C(x_i)\nonumber\\
\mathrm{subject \; to} &\nonumber \\
&\sum_{j=1}^n x_{ij} \le\mathrm{CAP}_i \; \forall j \in \{1,2,\dots,m\}\nonumber \\
&x_i \ge 0 \; \forall i \in \{1,2, \dots ,n\} \nonumber \\
\nonumber
\end{align}
Umbkaudu hinnates sisaldab see formulatsioon $225$ erinevat kombinatsiooni k\"utusest ja elektrijaamast, mis laotub laiali \"uhe aasta peale tunniajalistes vahemikes. Seega halvimal juhul peab optimeerimismudel suutma hallata ca. 1 971 000 muutjat.
Kulufunktsioon omab j\"argnevat \"uldist kuju
\begin{align}
C = \sum_{j=1}^m C_{const j} &+ \sum_{i=1}^n\sum_{j=1}^m C_{CO_2 ij}x_{ij} \nonumber \\
& + \sum_{i=1}^n\sum_{j=1}^m C_{SO_x ij}x_{ij} \nonumber \\
& + \sum_{i=1}^n\sum_{j=1}^m C_{N ij}x_{ij} \nonumber \\
& + \sum_{i=1}^n C_{f i}\sum_{j=1}^m x_{ij} \nonumber \\
\nonumber
\end{align}

Tulufunktsioon omab j\"argnevat \"uldist kuju
\begin{align}
P(x_i) &= \sum_{j=1}^m P_{ij}(x_{ij})\nonumber \\
\nonumber
\end{align}

Muutuja $x_{ij}$ t\"ahistab k\"utuse $i$ kasutusintensiivsust elektrijaamas $j$ m\~o\~odetuna \"uhikuga \emph{tonni tunnis}.
\section {Lahendusalgoritm}
Quadratic programming kujul
\begin{align}
\mathrm{min}\; Z = \sum\mathbf{z}_1 + \sum\mathbf{z}_2 &\nonumber \\
\mathrm{subject \; to}\qquad \qquad\qquad  &\nonumber\\
\mathbf{Qx} + \mathbf{A}^T\mathbf{u} - \mathbf{y} + \mathbf{z}_1 &= \mathbf{c}^T \nonumber\\
\mathbf{Ax} + \mathbf{v} - \mathbf{z}_2 &= \mathbf{b} \nonumber \\
\mathbf{x}^T\mathbf{y} + \mathbf{u}^T\mathbf{v} &= \mathbf{0} \nonumber \\
\mathbf{x} \ge 0, \quad\mathbf{u} \ge 0, & \quad\mathbf{y} \ge 0, \quad\mathbf{v} \ge 0 \nonumber \\
\nonumber
\end{align}
Alternatiivina, juhul kui mittelineaarsed funktsioonid on aproksimeeritavad t\"ukati lineaarsetena saaks kasutada separable programmingut. Samas sisaldab see l\"ahenemine potentsiaalset ohtu niigi suure muutujate arvu mitmekordistumises.
\end{document}